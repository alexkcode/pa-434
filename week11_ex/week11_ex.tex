\PassOptionsToPackage{unicode=true}{hyperref} % options for packages loaded elsewhere
\PassOptionsToPackage{hyphens}{url}
%
\documentclass[]{article}
\usepackage{lmodern}
\usepackage{amssymb,amsmath}
\usepackage{ifxetex,ifluatex}
\usepackage{fixltx2e} % provides \textsubscript
\ifnum 0\ifxetex 1\fi\ifluatex 1\fi=0 % if pdftex
  \usepackage[T1]{fontenc}
  \usepackage[utf8]{inputenc}
  \usepackage{textcomp} % provides euro and other symbols
\else % if luatex or xelatex
  \usepackage{unicode-math}
  \defaultfontfeatures{Ligatures=TeX,Scale=MatchLowercase}
\fi
% use upquote if available, for straight quotes in verbatim environments
\IfFileExists{upquote.sty}{\usepackage{upquote}}{}
% use microtype if available
\IfFileExists{microtype.sty}{%
\usepackage[]{microtype}
\UseMicrotypeSet[protrusion]{basicmath} % disable protrusion for tt fonts
}{}
\IfFileExists{parskip.sty}{%
\usepackage{parskip}
}{% else
\setlength{\parindent}{0pt}
\setlength{\parskip}{6pt plus 2pt minus 1pt}
}
\usepackage{hyperref}
\hypersetup{
            pdftitle={PA 434 Week11},
            pdfborder={0 0 0},
            breaklinks=true}
\urlstyle{same}  % don't use monospace font for urls
\usepackage[margin=1in]{geometry}
\usepackage{color}
\usepackage{fancyvrb}
\newcommand{\VerbBar}{|}
\newcommand{\VERB}{\Verb[commandchars=\\\{\}]}
\DefineVerbatimEnvironment{Highlighting}{Verbatim}{commandchars=\\\{\}}
% Add ',fontsize=\small' for more characters per line
\usepackage{framed}
\definecolor{shadecolor}{RGB}{248,248,248}
\newenvironment{Shaded}{\begin{snugshade}}{\end{snugshade}}
\newcommand{\AlertTok}[1]{\textcolor[rgb]{0.94,0.16,0.16}{#1}}
\newcommand{\AnnotationTok}[1]{\textcolor[rgb]{0.56,0.35,0.01}{\textbf{\textit{#1}}}}
\newcommand{\AttributeTok}[1]{\textcolor[rgb]{0.77,0.63,0.00}{#1}}
\newcommand{\BaseNTok}[1]{\textcolor[rgb]{0.00,0.00,0.81}{#1}}
\newcommand{\BuiltInTok}[1]{#1}
\newcommand{\CharTok}[1]{\textcolor[rgb]{0.31,0.60,0.02}{#1}}
\newcommand{\CommentTok}[1]{\textcolor[rgb]{0.56,0.35,0.01}{\textit{#1}}}
\newcommand{\CommentVarTok}[1]{\textcolor[rgb]{0.56,0.35,0.01}{\textbf{\textit{#1}}}}
\newcommand{\ConstantTok}[1]{\textcolor[rgb]{0.00,0.00,0.00}{#1}}
\newcommand{\ControlFlowTok}[1]{\textcolor[rgb]{0.13,0.29,0.53}{\textbf{#1}}}
\newcommand{\DataTypeTok}[1]{\textcolor[rgb]{0.13,0.29,0.53}{#1}}
\newcommand{\DecValTok}[1]{\textcolor[rgb]{0.00,0.00,0.81}{#1}}
\newcommand{\DocumentationTok}[1]{\textcolor[rgb]{0.56,0.35,0.01}{\textbf{\textit{#1}}}}
\newcommand{\ErrorTok}[1]{\textcolor[rgb]{0.64,0.00,0.00}{\textbf{#1}}}
\newcommand{\ExtensionTok}[1]{#1}
\newcommand{\FloatTok}[1]{\textcolor[rgb]{0.00,0.00,0.81}{#1}}
\newcommand{\FunctionTok}[1]{\textcolor[rgb]{0.00,0.00,0.00}{#1}}
\newcommand{\ImportTok}[1]{#1}
\newcommand{\InformationTok}[1]{\textcolor[rgb]{0.56,0.35,0.01}{\textbf{\textit{#1}}}}
\newcommand{\KeywordTok}[1]{\textcolor[rgb]{0.13,0.29,0.53}{\textbf{#1}}}
\newcommand{\NormalTok}[1]{#1}
\newcommand{\OperatorTok}[1]{\textcolor[rgb]{0.81,0.36,0.00}{\textbf{#1}}}
\newcommand{\OtherTok}[1]{\textcolor[rgb]{0.56,0.35,0.01}{#1}}
\newcommand{\PreprocessorTok}[1]{\textcolor[rgb]{0.56,0.35,0.01}{\textit{#1}}}
\newcommand{\RegionMarkerTok}[1]{#1}
\newcommand{\SpecialCharTok}[1]{\textcolor[rgb]{0.00,0.00,0.00}{#1}}
\newcommand{\SpecialStringTok}[1]{\textcolor[rgb]{0.31,0.60,0.02}{#1}}
\newcommand{\StringTok}[1]{\textcolor[rgb]{0.31,0.60,0.02}{#1}}
\newcommand{\VariableTok}[1]{\textcolor[rgb]{0.00,0.00,0.00}{#1}}
\newcommand{\VerbatimStringTok}[1]{\textcolor[rgb]{0.31,0.60,0.02}{#1}}
\newcommand{\WarningTok}[1]{\textcolor[rgb]{0.56,0.35,0.01}{\textbf{\textit{#1}}}}
\usepackage{graphicx,grffile}
\makeatletter
\def\maxwidth{\ifdim\Gin@nat@width>\linewidth\linewidth\else\Gin@nat@width\fi}
\def\maxheight{\ifdim\Gin@nat@height>\textheight\textheight\else\Gin@nat@height\fi}
\makeatother
% Scale images if necessary, so that they will not overflow the page
% margins by default, and it is still possible to overwrite the defaults
% using explicit options in \includegraphics[width, height, ...]{}
\setkeys{Gin}{width=\maxwidth,height=\maxheight,keepaspectratio}
\setlength{\emergencystretch}{3em}  % prevent overfull lines
\providecommand{\tightlist}{%
  \setlength{\itemsep}{0pt}\setlength{\parskip}{0pt}}
\setcounter{secnumdepth}{0}
% Redefines (sub)paragraphs to behave more like sections
\ifx\paragraph\undefined\else
\let\oldparagraph\paragraph
\renewcommand{\paragraph}[1]{\oldparagraph{#1}\mbox{}}
\fi
\ifx\subparagraph\undefined\else
\let\oldsubparagraph\subparagraph
\renewcommand{\subparagraph}[1]{\oldsubparagraph{#1}\mbox{}}
\fi

% set default figure placement to htbp
\makeatletter
\def\fps@figure{htbp}
\makeatother


\title{PA 434 Week11}
\author{}
\date{\vspace{-2.5em}}

\begin{document}
\maketitle

\hypertarget{loop}{%
\section{1. Loop}\label{loop}}

\begin{Shaded}
\begin{Highlighting}[]
\NormalTok{output_}\DecValTok{1}\NormalTok{ <-}\StringTok{ }\KeywordTok{vector}\NormalTok{(}\StringTok{"numeric"}\NormalTok{, }\DecValTok{10}\NormalTok{)}
\CommentTok{# since our sequence should be the length of our vector}
\CommentTok{# we should make the contruction of our loop based on the length directly}
\ControlFlowTok{for}\NormalTok{ (i }\ControlFlowTok{in} \KeywordTok{seq}\NormalTok{(}\KeywordTok{length}\NormalTok{(output_}\DecValTok{1}\NormalTok{)))\{}
\NormalTok{  output_}\DecValTok{1}\NormalTok{[[i]] <-}\StringTok{ }\NormalTok{i }\OperatorTok{+}\StringTok{ }\DecValTok{6}
\NormalTok{\}}
\NormalTok{output_}\DecValTok{1}
\end{Highlighting}
\end{Shaded}

\begin{verbatim}
##  [1]  7  8  9 10 11 12 13 14 15 16
\end{verbatim}

\hypertarget{matrix}{%
\section{2. Matrix}\label{matrix}}

\begin{Shaded}
\begin{Highlighting}[]
\NormalTok{mat_x <-}\StringTok{ }\KeywordTok{matrix}\NormalTok{(}\DataTypeTok{data =} \DecValTok{1}\OperatorTok{:}\DecValTok{120}\NormalTok{, }\DataTypeTok{nrow =} \DecValTok{20}\NormalTok{, }\DataTypeTok{ncol =} \DecValTok{6}\NormalTok{)}
\end{Highlighting}
\end{Shaded}

\hypertarget{a.}{%
\subsubsection{a.}\label{a.}}

\begin{Shaded}
\begin{Highlighting}[]
\CommentTok{# create the vector to store our output}
\NormalTok{output_2a <-}\StringTok{ }\KeywordTok{vector}\NormalTok{(}\StringTok{"numeric"}\NormalTok{, }\KeywordTok{ncol}\NormalTok{(mat_x))}
\CommentTok{# get the number of columns of the matrix and use that as the basis for the length of the loop}
\ControlFlowTok{for}\NormalTok{ (j }\ControlFlowTok{in} \KeywordTok{seq}\NormalTok{(}\KeywordTok{ncol}\NormalTok{(mat_x))) \{}
\NormalTok{  output_2a[[j]] <-}\StringTok{ }\KeywordTok{sum}\NormalTok{(mat_x[,j])}
\NormalTok{\}}
\NormalTok{output_2a}
\end{Highlighting}
\end{Shaded}

\begin{verbatim}
## [1]  210  610 1010 1410 1810 2210
\end{verbatim}

\hypertarget{b.}{%
\subsubsection{b.}\label{b.}}

\begin{Shaded}
\begin{Highlighting}[]
\CommentTok{# setting margin to 2 tells the apply function to apply the given function on the columns}
\NormalTok{output_2b <-}\StringTok{ }\KeywordTok{apply}\NormalTok{(}\DataTypeTok{X =}\NormalTok{ mat_x, }\DataTypeTok{FUN =}\NormalTok{ sum, }\DataTypeTok{MARGIN =} \DecValTok{2}\NormalTok{)}
\NormalTok{output_2b}
\end{Highlighting}
\end{Shaded}

\begin{verbatim}
## [1]  210  610 1010 1410 1810 2210
\end{verbatim}

\hypertarget{data-frame}{%
\section{3. Data Frame}\label{data-frame}}

\begin{Shaded}
\begin{Highlighting}[]
\NormalTok{df <-}\StringTok{ }\KeywordTok{data.frame}\NormalTok{(}\DecValTok{1}\OperatorTok{:}\DecValTok{10}\NormalTok{, }\KeywordTok{c}\NormalTok{(letters[}\DecValTok{1}\OperatorTok{:}\DecValTok{10}\NormalTok{]), }\KeywordTok{rnorm}\NormalTok{(}\DecValTok{10}\NormalTok{, }\DataTypeTok{sd =} \DecValTok{10}\NormalTok{), }\DataTypeTok{stringsAsFactors =} \OtherTok{FALSE}\NormalTok{)}
\CommentTok{# iterate over the columns of the df as objects}
\CommentTok{# this puts the column in the namespace of each iteration of the loop so that it may be}
\CommentTok{# treated a normal variable }
\ControlFlowTok{for}\NormalTok{ (col }\ControlFlowTok{in}\NormalTok{ df) \{}
  \ControlFlowTok{if}\NormalTok{ (}\KeywordTok{is.numeric}\NormalTok{(col)) \{}
    \KeywordTok{print}\NormalTok{(}\KeywordTok{mean}\NormalTok{(col))}
\NormalTok{  \} }\ControlFlowTok{else} \ControlFlowTok{if}\NormalTok{ (}\KeywordTok{is.character}\NormalTok{(col)) \{}
    \KeywordTok{print}\NormalTok{(}\KeywordTok{length}\NormalTok{(col))}
\NormalTok{  \} }\ControlFlowTok{else}\NormalTok{ \{}
    \KeywordTok{print}\NormalTok{(}\StringTok{"column is not numeric or character like"}\NormalTok{)}
\NormalTok{  \}}
\NormalTok{\}}
\end{Highlighting}
\end{Shaded}

\begin{verbatim}
## [1] 5.5
## [1] 10
## [1] 2.713929
\end{verbatim}

\hypertarget{matrix-of-distributions}{%
\section{4. Matrix of Distributions}\label{matrix-of-distributions}}

\begin{Shaded}
\begin{Highlighting}[]
\NormalTok{mat_distributions <-}\StringTok{ }\KeywordTok{matrix}\NormalTok{(}\DataTypeTok{nrow =} \DecValTok{10}\NormalTok{, }\DataTypeTok{ncol =} \DecValTok{4}\NormalTok{)}
\NormalTok{means <-}\StringTok{ }\KeywordTok{c}\NormalTok{(}\OperatorTok{-}\DecValTok{10}\NormalTok{, }\DecValTok{0}\NormalTok{, }\DecValTok{10}\NormalTok{, }\DecValTok{100}\NormalTok{)}
\CommentTok{# looping over means and mat_distributions in parallel }
\CommentTok{# with means providing the mean paramenter for the rnorm function}
\CommentTok{# to generate columns for mat_distributions}
\ControlFlowTok{for}\NormalTok{ (j }\ControlFlowTok{in} \KeywordTok{seq}\NormalTok{(}\KeywordTok{ncol}\NormalTok{(mat_distributions))) \{}
\NormalTok{  mat_distributions[,j] <-}\StringTok{ }\KeywordTok{rnorm}\NormalTok{(}\DecValTok{10}\NormalTok{, }\DataTypeTok{mean =}\NormalTok{ means[j])}
\NormalTok{\}}
\NormalTok{mat_distributions}
\end{Highlighting}
\end{Shaded}

\begin{verbatim}
##             [,1]       [,2]      [,3]      [,4]
##  [1,]  -9.470415  0.2508031 12.549448 100.03676
##  [2,]  -9.717127  0.3343869 11.335376 100.74567
##  [3,]  -9.770691  0.7429913  7.439233  99.73404
##  [4,] -10.310307  0.4371832  8.132252  99.86665
##  [5,]  -9.482775 -0.4089525 10.086516  99.38162
##  [6,]  -9.391016  0.8852963  8.745570  99.80975
##  [7,]  -9.281781 -0.4741776 11.928594 100.54127
##  [8,]  -9.161001  1.4490712  9.994941 101.56648
##  [9,]  -8.465406 -0.7813746 10.019036  98.81321
## [10,]  -8.886374  0.6846253  9.754575  99.61525
\end{verbatim}

\hypertarget{ifelse}{%
\section{5. Ifelse}\label{ifelse}}

\begin{Shaded}
\begin{Highlighting}[]
\NormalTok{respondent.df =}\StringTok{ }\KeywordTok{data.frame}\NormalTok{(}
  \DataTypeTok{name =} \KeywordTok{c}\NormalTok{(}\StringTok{"Sue"}\NormalTok{, }\StringTok{"Eva"}\NormalTok{, }\StringTok{"Henry"}\NormalTok{, }\StringTok{"Jan"}\NormalTok{, }\StringTok{"Mary"}\NormalTok{, }\StringTok{"John"}\NormalTok{),}
  \DataTypeTok{sex =} \KeywordTok{c}\NormalTok{(}\StringTok{"f"}\NormalTok{, }\StringTok{"f"}\NormalTok{, }\StringTok{"m"}\NormalTok{, }\StringTok{"m"}\NormalTok{, }\StringTok{"f"}\NormalTok{, }\StringTok{"m"}\NormalTok{),}
  \DataTypeTok{years =} \KeywordTok{c}\NormalTok{(}\DecValTok{21}\NormalTok{, }\DecValTok{31}\NormalTok{, }\DecValTok{29}\NormalTok{, }\DecValTok{19}\NormalTok{, }\DecValTok{23}\NormalTok{, }\DecValTok{33}\NormalTok{)}
\NormalTok{)}

\CommentTok{# create new column and assign values based on sex and years columns}
\NormalTok{respondent.df}\OperatorTok{$}\NormalTok{male.teen <-}
\StringTok{  }\KeywordTok{ifelse}\NormalTok{(}\DataTypeTok{test =}\NormalTok{ respondent.df}\OperatorTok{$}\NormalTok{sex }\OperatorTok{==}\StringTok{ "m"} \OperatorTok{&}\StringTok{ }\NormalTok{respondent.df}\OperatorTok{$}\NormalTok{years }\OperatorTok{<}\StringTok{ }\DecValTok{20}\NormalTok{,}
         \DataTypeTok{yes =} \DecValTok{1}\NormalTok{,}
         \DataTypeTok{no =} \DecValTok{0}\NormalTok{)}
\NormalTok{respondent.df}
\end{Highlighting}
\end{Shaded}

\begin{verbatim}
##    name sex years male.teen
## 1   Sue   f    21         0
## 2   Eva   f    31         0
## 3 Henry   m    29         0
## 4   Jan   m    19         1
## 5  Mary   f    23         0
## 6  John   m    33         0
\end{verbatim}

\hypertarget{ifelse-versus-if_else}{%
\section{6. Ifelse versus If\_else}\label{ifelse-versus-if_else}}

\begin{Shaded}
\begin{Highlighting}[]
\KeywordTok{library}\NormalTok{(tidyverse)}

\NormalTok{respondent.df }\OperatorTok
\StringTok{  }\KeywordTok{mutate}\NormalTok{(}\DataTypeTok{under30 =} \KeywordTok{ifelse}\NormalTok{(years }\OperatorTok{>}\StringTok{ }\DecValTok{30}\NormalTok{, }\OtherTok{NA}\NormalTok{, years),}
         \DataTypeTok{under30.tidy =} \KeywordTok{if_else}\NormalTok{(years }\OperatorTok{>}\StringTok{ }\DecValTok{30}\NormalTok{, }\KeywordTok{as.double}\NormalTok{(}\OtherTok{NA}\NormalTok{), years))}
\end{Highlighting}
\end{Shaded}

\begin{verbatim}
##    name sex years male.teen under30 under30.tidy
## 1   Sue   f    21         0      21           21
## 2   Eva   f    31         0      NA           NA
## 3 Henry   m    29         0      29           29
## 4   Jan   m    19         1      19           19
## 5  Mary   f    23         0      23           23
## 6  John   m    33         0      NA           NA
\end{verbatim}

\hypertarget{tapply-mean-min-and-max}{%
\section{7. tapply mean, min and max}\label{tapply-mean-min-and-max}}

\begin{Shaded}
\begin{Highlighting}[]
\ControlFlowTok{for}\NormalTok{ (fun }\ControlFlowTok{in} \KeywordTok{c}\NormalTok{(}\StringTok{"mean"}\NormalTok{, }\StringTok{"min"}\NormalTok{, }\StringTok{"max"}\NormalTok{)) \{}
  \CommentTok{# print the name of the function that we iterated to}
  \KeywordTok{print}\NormalTok{(fun)}
  \CommentTok{# apply function with the name given by variable "fun" across the sexes }
  \KeywordTok{print}\NormalTok{(}\KeywordTok{tapply}\NormalTok{(}\DataTypeTok{X =}\NormalTok{ respondent.df}\OperatorTok{$}\NormalTok{years, }\DataTypeTok{INDEX =}\NormalTok{ respondent.df}\OperatorTok{$}\NormalTok{sex, }\DataTypeTok{FUN =} \KeywordTok{get}\NormalTok{(fun)))}
\NormalTok{\}}
\end{Highlighting}
\end{Shaded}

\begin{verbatim}
## [1] "mean"
##  f  m 
## 25 27 
## [1] "min"
##  f  m 
## 21 19 
## [1] "max"
##  f  m 
## 31 33
\end{verbatim}

\hypertarget{tidy}{%
\section{8. Tidy}\label{tidy}}

\begin{Shaded}
\begin{Highlighting}[]
\NormalTok{author =}\StringTok{ }\KeywordTok{c}\NormalTok{(}
  \StringTok{"Author1"}\NormalTok{,}
  \StringTok{"Author1"}\NormalTok{,}
  \StringTok{"Author2"}\NormalTok{,}
  \StringTok{"Author3"}\NormalTok{,}
  \StringTok{"Author3"}\NormalTok{,}
  \StringTok{"Author3"}\NormalTok{,}
  \StringTok{"Author4"}\NormalTok{,}
  \StringTok{"Author5"}
\NormalTok{)}
\NormalTok{pub =}\StringTok{ }\KeywordTok{c}\NormalTok{(}\StringTok{"Pub1"}\NormalTok{, }\StringTok{"Pub2"}\NormalTok{, }\StringTok{"Pub3"}\NormalTok{, }\StringTok{"Pub4"}\NormalTok{,}
        \StringTok{"Pub5"}\NormalTok{, }\StringTok{"Pub6"}\NormalTok{, }\StringTok{"Pub7"}\NormalTok{, }\StringTok{"Pub8"}\NormalTok{)}
\NormalTok{type =}\StringTok{ }\KeywordTok{c}\NormalTok{(}
  \StringTok{"preprint"}\NormalTok{,}
  \StringTok{"article"}\NormalTok{,}
  \StringTok{"preprint"}\NormalTok{,}
  \StringTok{"article"}\NormalTok{,}
  \StringTok{"article"}\NormalTok{,}
  \StringTok{"preprint"}\NormalTok{,}
  \StringTok{"preprint"}\NormalTok{,}
  \StringTok{"article"}
\NormalTok{)}
\NormalTok{data <-}\StringTok{ }\KeywordTok{as_tibble}\NormalTok{(}\KeywordTok{cbind}\NormalTok{(author, pub, type))}
\KeywordTok{print}\NormalTok{(data)}
\end{Highlighting}
\end{Shaded}

\begin{verbatim}
## # A tibble: 8 x 3
##   author  pub   type    
##   <chr>   <chr> <chr>   
## 1 Author1 Pub1  preprint
## 2 Author1 Pub2  article 
## 3 Author2 Pub3  preprint
## 4 Author3 Pub4  article 
## 5 Author3 Pub5  article 
## 6 Author3 Pub6  preprint
## 7 Author4 Pub7  preprint
## 8 Author5 Pub8  article
\end{verbatim}

\begin{Shaded}
\begin{Highlighting}[]
\CommentTok{# split into groups based on author}
\NormalTok{data.list <-}\StringTok{ }\KeywordTok{split}\NormalTok{(data, author)}

\CommentTok{# Create a loop that will number the publications for each authors. }
\CommentTok{# “Tidy” the data so that each row represents one author only.}
\NormalTok{num_pub =}\StringTok{ }\KeywordTok{vector}\NormalTok{(}\StringTok{"numeric"}\NormalTok{, }\KeywordTok{length}\NormalTok{(data))}

\ControlFlowTok{for}\NormalTok{ (i }\ControlFlowTok{in} \DecValTok{1}\OperatorTok{:}\NormalTok{(}\KeywordTok{length}\NormalTok{(data.list))) \{}
\NormalTok{\}}
\end{Highlighting}
\end{Shaded}

\end{document}
